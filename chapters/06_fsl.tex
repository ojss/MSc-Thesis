\chapter{Few-Shot Learning}\label{chap:fsl}
A good machine learning model often requires training with a large number of samples. Humans, on the contrary, learn new concepts and skills much faster and more efficiently. Children who have only seen cats and birds a few times can quickly tell them apart. People who know how to ride a bike are likely to discover how to ride a motorcycle quickly with little or even no demonstration. Is it possible to design a machine learning model with similar properties - learning new concepts and skills fast with a few training examples? That is essentially what few-shot learning is designed to solve.

Despite notable advances in the field of artificial intelligence, two essential aspects of human conceptual intelligence have consistently eluded machine learning and artificial intelligence (AI) systems. 
First, for most interesting kinds of natural and man-made categories of entities, humans can learn a new concept from just one or a few handful examples, whereas many AI models would require several thousands of examples to perform satisfactorily. 
Second, people learn far richer representations than machines do, even for seemingly simple concepts, and use them for a wide variety of tasks such as creating new entities based on the exemplars, classifying objects into parts, grasping between concepts and parts, and creating new abstract categories (concepts) by combining existing ideas and concepts.
In contrast to this, the best machine learning models and neural networks cannot perform these additional functions using their specialised learnt representations. 
The challenge arises when we wish for AI models to learn new concepts and representations from few examples and ensure that these representations are abstract and flexible.

